\documentclass[a4paper, twocolumn]{article}
\usepackage[utf8]{inputenc}
\usepackage{geometry} 
\usepackage{ctex}
\usepackage{amsmath}
\usepackage{amssymb}
\usepackage{amsthm}
\usepackage{listings}
\usepackage{xcolor}
\usepackage{hyperref} 
\usepackage{booktabs}
\usepackage[numbers]{natbib}
\usepackage{graphicx}
\usepackage{subfigure}
\usepackage{makecell}
\usepackage{float}
\usepackage{url}
\usepackage{tikz}


\usetikzlibrary{arrows, positioning, shapes, arrows.meta, positioning, shapes.geometric,calc}

\setlength{\floatsep}{5pt plus 2pt minus 2pt} % 调整浮动体之间的间距
\setlength{\textfloatsep}{5pt plus 2pt minus 2pt} % 调整文本与浮动体之间的间距
\setlength{\intextsep}{5pt plus 2pt minus 2pt} % 调整文本中的浮动体间距

\geometry{margin=1.7cm, columnsep=20pt} 
\lstset{
    basicstyle=\ttfamily\footnotesize,
    breaklines=true,
    frame=single,
    numbers=left,
    numbersep=10pt,
    xleftmargin=1em,
    xrightmargin=1em,
    keywordstyle=\color{blue},
    commentstyle=\color{gray},
}

\title{课程报告一:手搓MLP在MNIST及Fashion-MNIST上不同超参数的对照实验}
\author{薛亦呈}
\date{\today}

\begin{document}
\maketitle

\section{问题简介}

本实验旨在通过手动实现多层感知机(MLP),深入探究不同参数配置对分类任务性能的影响。
基于反向传播算法,分别在 MNIST、Fashion-MNIST 等数据集上进行训练与测试,系统分析网络结构、激活函数、优化器等因素的作用机制,
并尝试通过多种优化策略提升程序性能。研究结果将为深度学习模型的参数选择和性能优化提供实践指导,同时增强对神经网络工作原理的理解。

\section{模型与算法介绍}

\subsection{反向传播算法的计算图表示}
反向传播通过计算图实现梯度的高效计算:
反向传播通过计算图实现梯度的高效计算,其核心思想是利用链式法则逐层计算梯度\cite{Goodfellow-et-al-2016}。
计算图由以下部分组成:
\begin{itemize}
    \item \textbf{节点}: 表示运算 (如矩阵运算、激活函数)
    \item \textbf{边}: 表示数据流向 (如输入特征、权重矩阵)
    \item \textbf{前向传播}: 沿计算图计算各层输出值
    \item \textbf{反向传播}: 从损失函数出发,利用链式法则逐层计算梯度 ($\frac{\partial L}{\partial W}$),更新权重
\end{itemize} 

\begin{figure}[H]
    \centering
    \resizebox{\columnwidth}{!}{
    \begin{tikzpicture}[
        node distance=2cm,
        op/.style={rectangle, draw=blue!50, fill=blue!10, thick, minimum size=8mm},
        var/.style={circle, draw=black!50, fill=green!10, thick, minimum size=8mm},
        grad/.style={red, dashed, -latex}
    ]
      
      
    % 前向传播节点
    \node[var] (X) {\( \mathbf{x} \)};
    \node[op, right=of X] (W) {MatMul};
    \node[op, right=of W] (sigma) {ReLU};
    \node[var, right=of sigma] (L) {\( \mathcal{L} \)};
    \node[var, below=of W] (W_val) {\( \mathbf{W} \)};
    
    % 前向传播连接
    \draw[->] (X) -- node[above] {输入} (W);
    \draw[->] (W_val) -- node[right] {权重} (W);
    \draw[->] (W) -- node[above] {\(\mathbf{z}\)} (sigma);
    \draw[->] (sigma) -- node[above] {\(\mathbf{a}\)} (L);
    
    % 反向传播连接
    \draw[grad] (sigma) to[bend left=20] node[below, sloped] {\(\frac{\partial \mathcal{L}}{\partial \mathbf{z}}\)} (W);
    \draw[grad] (L) to[bend left=20] node[below, sloped] {\(\frac{\partial \mathcal{L}}{\partial \mathbf{a}}\)} (sigma);
    \draw[grad] (W.240) -- node[below, sloped] {\( \frac{\partial \mathcal{L}}{\partial \mathbf{W}} \)} (W_val.north);
    \draw[grad] (W.200) -- node[below, sloped] {\( \frac{\partial \mathcal{L}}{\partial \mathbf{x}} \)} (X.east);

    \end{tikzpicture}
    }
    \caption{反向传播计算图(红色虚线为梯度传播路径)}
    \label{fig:backprop}
\end{figure}

这种机制使得神经网络能够在复杂的非线性映射中找到最优解。


\subsection{网络结构与关键组件}
本实验采用的网络架构如下:
\begin{itemize}
    \item \textbf{网络架构}: FullyConnected (Exp: [784, 256, 64, 10])
    \item \textbf{激活函数}: Sigmoid、ReLU、Tanh
    \item \textbf{正则化}: Dropout
    \item \textbf{优化器}: SGD 、 Adam
    \item \textbf{损失函数}: CrossEntropy
\end{itemize}

\begin{figure}[H]
    \centering
    \resizebox{\columnwidth}{!}{
    \begin{tikzpicture}[
        neuron/.style={circle, draw=black, fill=white, minimum size=5.2mm},
        arrow/.style={-{stealth[scale=0.6]}, semithick, black},  % 恢复箭头尺寸
        x=3.3cm,
        layer label/.style={below=8mm, font=\footnotesize\bfseries}
    ]

    % ===== 统一间距参数 =====
    \def\spacing{0.6}

    % ===== 各层定义 =====
    % 输入层 (6 neurons)
    \foreach \i [evaluate=\i as \y using (3.5-\i)*\spacing] in {1,...,6}
        \node[neuron] (in-\i) at (0,\y) {};

    % 隐藏层1 (5 neurons)
    \foreach \i [evaluate=\i as \y using (3-\i)*\spacing] in {1,...,5}
        \node[neuron] (h1-\i) at (1,\y) {};

    % 隐藏层2 (4 neurons)
    \foreach \i [evaluate=\i as \y using (2.5-\i)*\spacing] in {1,...,4}
        \node[neuron] (h2-\i) at (2,\y) {};

    % 输出层 (3 neurons)
    \foreach \i [evaluate=\i as \y using (2-\i)*\spacing] in {1,...,3}
        \node[neuron] (out-\i) at (3,\y) {};

    % ===== 连接线优化 =====
    \node[right=20mm of out-2] (L) {\textit{\large\(\mathcal{L}\)}};
    \draw[arrow] (out-2) -- 
        node[above, sloped, font=\footnotesize, pos=0.5] 
        {CrossEntropy} (L);

    \foreach \i in {1,...,6} \foreach \j in {1,...,5} 
        \draw[arrow] (in-\i) -- (h1-\j);
    \foreach \i in {1,...,5} \foreach \j in {1,...,4} 
        \draw[arrow] (h1-\i) -- (h2-\j);
    \foreach \i in {1,...,4} \foreach \j in {1,...,3} 
        \draw[arrow] (h2-\i) -- (out-\j);

    % ===== ReLU标注修正 =====
    \draw[orange!80!black, -{stealth[scale=0.6]}, thick] 
        ($(h2-2.north east)+(1mm,1mm)$) -- ++(45:6mm)  % 调整起点和角度
        node[right=1mm, font=\footnotesize] (r1) {ReLU};

    \draw[orange!80!black, -{stealth[scale=0.6]}, thick] 
        ($(h1-2.north east)+(0.3mm,0.3mm)$) -- ++(45:6mm)  % 调整起点和角度
        node[right=0.3mm, font=\footnotesize] (r2) {ReLU};
    
    \draw[orange!80!black, -{stealth[scale=0.6]}, thick] 
        ($(r2.north east)+(0.3mm,0.3mm)$) -- ++(5:4mm)  % 调整起点和角度
        node[right=0.4mm, font=\footnotesize] {Dropout};

    % ===== 层标签 =====
    \path (0,-1.6) node[layer label] {输入层}
          (1,-1.6) node[layer label] {隐藏层1}
          (2,-1.6) node[layer label] {隐藏层2}
          (3,-1.6) node[layer label] {输出层};

    \end{tikzpicture}
    }
    \caption{MLP网络架构示意图}
    \label{fig:mlp}
\end{figure}

\section{数据集与实验参数}

\subsection{数据集简介}
本实验采用的数据集如下:
\begin{itemize}
    \item \textbf{MNIST}: 手写数字灰度图(28×28),训练集60k,测试集10k
    \item \textbf{FashionMNIST}: 10类服饰灰度图(28×28),训练集60k,测试集10k
\end{itemize}

\subsection{实验配置}
表 1 展示了实验参数配置,包括网络层结构、激活函数、优化器和训练轮数等。
\begin{table}[H]
    \centering
    \caption{实验参数配置}
    \label{tab:config}
    \resizebox{\columnwidth}{!}{
    \begin{tabular}{@{}ll@{}}
        \toprule
        参数 & 选项 \\ 
        \midrule
        网络层结构 & \makecell[l]{{[784, 28, 10]}、{[784, 256, 128, 10]} \\ {[784, 512, 256, 128, 64, 10]}}\\
        激活函数 & Sigmoid、ReLU、Tanh\\
        优化器 & SGD (lr=0.01, batch=64)、Adam (lr=0.00035) \\
        训练轮数 & 30 epochs \\
        \bottomrule
    \end{tabular}
    }
\end{table}

\section{实验结果与分析}
本实验采用单一变量法,测试了不同网络深度、不同激活函数和不同优化器对模型性能的影响。
通过对比实验结果,分析了各因素对模型性能的影响,为深度学习模型的参数选择和性能优化提供实践指导。
\subsection{不同网络深度对比}
统一使用 ReLU 作为激活函数,Adam 作为优化器,对比不同网络深度在 MNIST 和 Fashion-MNIST 数据集上的表现。
表 2 展示了不同网络深度的准确率,图 1 至图 3 分别展示了浅层、中层和深层网络在两个数据集上的表现。
\begin{table}[H]
    \centering
    \caption{不同网络深度对比}
    \label{tab:netdepth}
    \resizebox{\columnwidth}{!}{
    \begin{tabular}{@{}lcc@{}}
        \toprule
        网络深度 & MNIST准确率 & FashionMNIST准确率 \\ 
        \midrule
        {[784, 28, 10]} & 96.8\% & 88.1\% \\
        {[784, 256, 128, 10]} & 98.3\% & 89.7\% \\
        {[784, 512, 256, 128, 64, 10]} & 98.4\% & 89.4\%  \\
        \bottomrule
    \end{tabular}
    }
\end{table}

\begin{figure}[H]
    \centering
    \subfigure[浅层网络在MNIST上的表现]{
        \includegraphics[width=0.45\textwidth]{image/11m.png}
        \label{fig:11m}
    }
    \subfigure[浅层网络在FashionMNIST上的表现]{
        \includegraphics[width=0.45\textwidth]{image/11f.png}
        \label{fig:11f}
    }
    \caption{浅层网络表现}
    \label{fig:11}
\end{figure}

\begin{figure}[H]
    \centering
    \subfigure[中层网络在MNIST上的表现]{
        \includegraphics[width=0.45\textwidth]{image/12m.png}
        \label{fig:12m}
    }
    \subfigure[中层网络在FashionMNIST上的表现]{
        \includegraphics[width=0.45\textwidth]{image/12f.png}
        \label{fig:12f}
    }
    \caption{中层网络表现}
    \label{fig:12}
\end{figure}

\begin{figure}[H]
    \centering
    \subfigure[深层网络在MNIST上的表现]{
        \includegraphics[width=0.45\textwidth]{image/13m.png}
        \label{fig:13m}
    }
    \subfigure[深层网络在FashionMNIST上的表现]{
        \includegraphics[width=0.45\textwidth]{image/13f.png}
        \label{fig:13f}
    }
    \caption{深层网络表现}
    \label{fig:13}
\end{figure}

\textbf{结论}:网络深度越深在 MNIST 上的表现越好,但随着深度加深这种优势逐渐减弱,并且在 FashionMNIST 上表现会略有退化。
              这与He的研究结果\cite{he2016deep}一致,深层网络在复杂数据集上可能出现过拟合或梯度消失问题。


\subsection{不同激活函数对比}
统一使用 [784, 256, 128, 10] 作为网络深度,Adam 作为优化器,对比不同激活函数在 MNIST 和 Fashion-MNIST 数据集上的表现。
表 3 展示了不同激活函数的准确率,图 4 至图 6 分别展示了 ReLU、Sigmoid 和 Tanh 在两个数据集上的表现。
\begin{table}[H]
    \centering
    \caption{激活函数性能对比}
    \label{tab:activation}
    \resizebox{\columnwidth}{!}{
    \begin{tabular}{@{}lcc@{}}
        \toprule
        激活函数 & MNIST准确率 & FashionMNIST准确率 \\ 
        \midrule
        ReLU & 98.2\% & 90.2\% \\
        Sigmoid & 97.8\% & 88.9\% \\
        Tanh & 97.7\% & 89.1\%  \\
        \bottomrule
    \end{tabular}
    }
\end{table}

\begin{figure}[H]
    \centering
    \subfigure[ReLU在MNIST上的表现]{
        \includegraphics[width=0.45\textwidth]{image/21m.png}
        \label{fig:21m}
    }
    \subfigure[ReLU在FashionMNIST上的表现]{
        \includegraphics[width=0.45\textwidth]{image/21f.png}
        \label{fig:21f}
    }
    \caption{ReLU表现}
    \label{fig:21}
\end{figure}

\begin{figure}[H]
    \centering
    \subfigure[Sigmoid在MNIST上的表现]{
        \includegraphics[width=0.45\textwidth]{image/22m.png}
        \label{fig:22m}
    }
    \subfigure[Sigmoid在FashionMNIST上的表现]{
        \includegraphics[width=0.45\textwidth]{image/22f.png}
        \label{fig:22f}
    }
    \caption{Sigmoid表现}
    \label{fig:22}
\end{figure}

\begin{figure}[H]
    \centering
    \subfigure[Tanh在MNIST上的表现]{
        \includegraphics[width=0.45\textwidth]{image/23m.png}
        \label{fig:23m}
    }
    \subfigure[Tanh在FashionMNIST上的表现]{
        \includegraphics[width=0.45\textwidth]{image/23f.png}
        \label{fig:23f}
    }
    \caption{Tanh表现}
    \label{fig:23}
\end{figure}

\textbf{结论}:ReLU 收敛更快且准确率更高,缓解梯度消失问题。
              Sigmoid 和 Tanh 在训练初期可能出现梯度消失,导致收敛速度较慢。

\subsection{不同优化器对比}
统一使用 [784, 256, 128, 10] 作为网络深度,ReLU 作为激活函数,对比不同优化器在 MNIST 和 Fashion-MNIST 数据集上的表现。
表 4 展示了不同优化器的准确率,图 7 和图 8 分别展示了 Adam 和 SGD 在两个数据集上的表现。
\begin{table}[H]
    \centering
    \caption{优化器性能对比}
    \label{tab:optimizer}
    \resizebox{\columnwidth}{!}{
    \begin{tabular}{@{}lcc@{}}
        \toprule
        优化器 & MNIST准确率 & FashionMNIST准确率 \\ 
        \midrule
        Adam(0.0035) & 98.4\% & 90.2\% \\
        SGD(0.01) & 82.4\% & 69.9\% \\
        \bottomrule
    \end{tabular}
    }
\end{table}

\begin{figure}[H]
    \centering
    \subfigure[Adam在MNIST上的表现]{
        \includegraphics[width=0.45\textwidth]{image/31m.png}
        \label{fig:31m}
    }
    \subfigure[Adam在FashionMNIST上的表现]{
        \includegraphics[width=0.45\textwidth]{image/31f.png}
        \label{fig:31f}
    }
    \caption{Adam表现}
    \label{fig:31}
\end{figure}

\begin{figure}[H]
    \centering
    \subfigure[SGD在MNIST上的表现]{
        \includegraphics[width=0.45\textwidth]{image/32m.png}
        \label{fig:32m}
    }
    \subfigure[SGD在FashionMNIST上的表现]{
        \includegraphics[width=0.45\textwidth]{image/32f.png}
        \label{fig:32f}
    }
    \caption{SGD表现}
    \label{fig:32}
\end{figure}

\textbf{结论}:Adam 结合动量与自适应学习率,综合性能更优,这与Kingma的研究结果\cite{kingma2015adam}一致。SGD 收敛速度较慢,且在复杂数据集上表现不佳。

\begin{figure}[H]
    \centering
    \includegraphics[width=0.45\textwidth]{image/HeatMap.png}
    \caption{不同参数配置下的准确率热力图}
    \label{fig:heatmap}
\end{figure}

\section{程序优化方案}
\begin{itemize}
    \item \textbf{批处理加速}:使用向量化运算替代循环\cite{hennessy2017computer}
    \item \textbf{内存管理}:限制中间变量存储,启用\lstinline{del}及时释放
    \item \textbf{并行计算}:利用PyTorch DataLoader多线程加载数据
    \item \textbf{早停法}:验证集损失连续3轮不下降时终止训练
\end{itemize}

经优化后,单epoch训练时间从10s缩短至5s(Macbook pro M3)。

\section{结论与心得}
通过本次实验,得出以下结论与心得:
\begin{itemize}
    \item 深层网络(如[784,256,64,10])较浅层网络在分类任务上提升约1-2\%的准确率,表明增加网络深度有助于学习更复杂的特征表示。
    \item Adam优化器在多数场景下表现稳定,推荐优先选用,其结合了动量和自适应学习率的优点,能够快速收敛并取得较好的性能。
    \item 程序优化需权衡代码可读性与执行效率,在追求性能提升的同时,应保持代码的清晰和可维护性。
    \item 未来的研究方向可以进一步探索更复杂的网络结构和优化算法,以提高模型在不同类型数据集上的泛化能力和鲁棒性。\cite{Goodfellow-et-al-2016}
\end{itemize}


\bibliographystyle{plain}
\bibliography{1.bib} 

\newpage

\section*{附录:代码实现}
关键代码片段 :
\noindent \textbf{基类定义}:
\begin{lstlisting}[language=Python]
#Based Class
from abc import ABC, abstractmethod

class Layer:
    def forward (self, x):
        raise NotImplementedError
    
    def backward (self, grad):
        raise NotImplementedError
    
    def train(self):
        pass
    
    def eval(self):
        pass
    
    def parameters(self):
        return []
            
class Optimizer(ABC):
    def __init__(self, parameters):
        self.parameters = parameters
    
    @abstractmethod
    def step(self):
        raise NotImplementedError("Optimizer must implement step()")
    
    def zero_grad(self):
         for param in self.parameters:
            if "dW" in param:
                param["dW"]().fill(0)
            if "db" in param:
                param["db"]().fill(0)

class Loss:
    def compute_loss(self, y_pred, y_true): 
        raise NotImplementedError
    
class DataSet(ABC):
    @abstractmethod
    def __len__(self):
        raise NotImplementedError
    
    @abstractmethod
    def __getitem__(self, index):
        raise NotImplementedError
    
    @abstractmethod
    def __iter__(self):
        for i in range(len(self)):
            yield self[i]
\end{lstlisting}

\noindent \textbf{激活函数实现}:
\begin{lstlisting}[language=Python]
#Activation Function Layer
class Sigmoid(Layer):
    def __init__(self):
        pass
    
    def sigmoid(self, x):
        return 1/(1+np.exp(-x))
    
    def forward(self, x):
        self.x = x
        self.y = self.sigmoid(self.x)
        return self.y
    
    def backward(self, d):
        sig = self.sigmoid(self.x)
        return d*sig*(1-sig)
\end{lstlisting}

\noindent \textbf{全连接层实现}:
\begin{lstlisting}[language=Python]
#Fully Connected Layer
class FullyConnected(Layer):
    def __init__(self, input_dim, output_dim):
        self.W = np.random.randn(output_dim, input_dim) * np.sqrt(2.0 / input_dim)  
        self.b = np.zeros((output_dim, 1))
        self.dW = np.zeros_like(self.W)
        self.db = np.zeros_like(self.b)
        self.x = None

    def forward(self, x):
        self.x = x  # x: (batch_size, input_dim)
        return np.dot(x, self.W.T) + self.b.T  # output: (batch_size, output_dim)

    def backward(self, d):
        # d: (batch_size, output_dim)
        self.dW = np.dot(d.T, self.x) / d.shape[0]  # (output_dim, input_dim)
        self.db = np.sum(d, axis=0, keepdims=True).T / d.shape[0]  # (output_dim, 1)
        return np.dot(d, self.W)  # (batch_size, input_dim)

    def parameters(self):
        return [{
            "W": lambda: self.W, 
            "b": lambda: self.b, 
            "dW": lambda: self.dW, 
            "db": lambda: self.db,
            "set_W": lambda new_W: setattr(self, "W", new_W), 
            "set_b": lambda new_b: setattr(self, "b", new_b)  
        }]

    def zero_grad(self):
        self.dW.fill(0)  
        self.db.fill(0)
\end{lstlisting}

\noindent \textbf{损失函数实现}:
\begin{lstlisting}[language=Python]
#Loss Function
class CrossEntropy(Loss):
    @staticmethod
    def compute_loss(y_pred, y_true):
        epsilon = 1e-15
        y_pred = np.clip(y_pred, epsilon, 1 - epsilon)
        
        loss = -np.mean(np.sum(y_true * np.log(y_pred), axis=1))
        grad = (y_pred - y_true) / y_pred.shape[0] 
        return loss, grad
\end{lstlisting}

\noindent \textbf{正则化层实现}:
\begin{lstlisting}[language=Python]
#Normalization Layer
class Dropout(Layer):
    def __init__(self, p=0.5):
        self.p = p
        self.mask = None
        self.is_training = False
        
    def forward(self, x):
        self.mask = np.random.rand(*x.shape) < (1 - self.p) 
        if  self.is_training:
            return x * self.mask /  (1 - self.p)
        else:
            return x
    
    def backward(self, grad):
        if self.is_training:
            return grad * self.mask
        else:
            return grad
    
    def train(self):
        self.is_training = True
        
    def eval(self):
        self.is_training = False
\end{lstlisting}

\noindent \textbf{优化器实现}:
\begin{lstlisting}[language=Python]
#Optimizer
class SGD(Optimizer):
    def __init__(self, parameters, lr=0.01):
        super().__init__(parameters)
        self.lr = lr

    def step(self):
        for param in self.parameters:
            for key in ['W', 'b']:
                g = param[f'd{key}']()  
                new_value = param[key]() - self.lr * g 
                param[f"set_{key}"](new_value)  

class Adam(Optimizer):
    def __init__(self, parameters, lr=0.001, beta1=0.9, beta2=0.999, epsilon=1e-8):
        super().__init__(parameters)
        self.lr = lr
        self.beta1 = beta1
        self.beta2 = beta2
        self.epsilon = epsilon
        self.m = [{k: np.zeros_like(param[k]) for k in ['W', 'b']} for param in parameters]
        self.v = [{k: np.zeros_like(param[k]) for k in ['W', 'b']} for param in parameters]
        self.t = 0

    def step(self):
        self.t += 1
        for i, param in enumerate(self.parameters):
            for key in ['W', 'b']:
                g = param[f'd{key}']()  
                self.m[i][key] = self.beta1 * self.m[i][key] + (1 - self.beta1) * g
                self.v[i][key] = self.beta2 * self.v[i][key] + (1 - self.beta2) * (g ** 2)
                m_hat = self.m[i][key] / (1 - self.beta1 ** self.t)
                v_hat = self.v[i][key] / (1 - self.beta2 ** self.t)
    
                new_value = param[key]() - self.lr * m_hat / (np.sqrt(v_hat) + self.epsilon)  
                param[f"set_{key}"](new_value)  
\end{lstlisting}

\noindent \textbf{网络类实现}:
\begin{lstlisting}[language=Python]
#Network
class Net:
    def __init__(self):
        self.layers = []
    
    def add(self, layer):
        self.layers.append(layer)
        
    def forward(self, x):
        for layer in self.layers:
            x = layer.forward(x)
        return x
    
    def backward(self, grad):
        for layer in reversed(self.layers):
            grad = layer.backward(grad)
        return grad
    
    def parameters(self):
        params = []
        for layer in self.layers:
            params += layer.parameters()
        return params
    
    def train(self):
        for layer in self.layers:
            layer.train()
    
    def eval(self):
        for layer in self.layers:
            layer.eval()
\end{lstlisting}

\noindent \textbf{DataSet、DataLoader实现}:
\begin{lstlisting}[language=Python]
#DataSet and DataLoader
class TensorDataSet(DataSet):
    def __init__(self, X, y=None, transform=None):
        self.X = X
        self.y = y
        self.transform = transform
    
    def __len__(self):
        return len(self.X)
    
    def __getitem__(self, idx):
        x = self.X[idx]
        if self.y is not None:
            y = self.y[idx]
        else:
            y = None
        
        if self.transform is not None:
            x = self.transform(x)
        
        return (x, y) if y is not None else x
    
    def __iter__(self):
        for i in range(len(self)):
            yield self[i]
    
class DataLoader:
    def __init__(self, dataset, batch_size=1, shuffle=False):
        self.dataset = dataset
        self.batch_size = batch_size
        self.shuffle = shuffle
        self.indices = np.arange(len(dataset))
        
    def __iter__(self):
        if self.shuffle:
            np.random.shuffle(self.indices)
            
        for i in range(0, len(self.dataset), self.batch_size):
            batch_indices = self.indices[i:i+self.batch_size]
            batch = [self.dataset[idx] for idx in batch_indices]
            
            x_batch = np.stack([item[0] for item in batch])
            y_batch = np.stack([item[1] for item in batch])
            
            yield (x_batch, y_batch) if y_batch is not None else x_batch

    def __len__(self):
        return (len(self.dataset) + self.batch_size - 1) // self.batch_size
\end{lstlisting}

\noindent \textbf{训练函数}:
\begin{lstlisting}[language=Python]
#Training Function
import time

def train(model, train_loader, test_loader, criterion, optimizer, epochs, plot=True):
    train_losses = []
    test_losses = []
    train_accs = []
    test_accs = []
    
    for epoch in range(epochs):
        model.train()
        train_loss = 0.0
        train_acc = 0.0
        for img, label in train_loader:
            optimizer.zero_grad()
            output = model.forward(img)
            loss, grad = criterion.compute_loss(output, label)
            model.backward(grad)
            optimizer.step()
            train_loss += loss
            train_acc += Metrics.accuracy(output, np.argmax(label, axis=1))
            
        train_loss /= len(train_loader)
        train_acc /= len(train_loader)

        model.eval()
        test_loss = 0.0
        test_acc = 0.0
        for img, label in test_loader:
            output = model.forward(img)
            loss, _ = criterion.compute_loss(output, label)
            test_loss += loss
            test_acc += Metrics.accuracy(output, np.argmax(label, axis=1))
        test_loss /= len(test_loader)
        test_acc /= len(test_loader)

        print(f"Epoch {epoch+1}/{epochs} | "
              f"Train Loss: {train_loss:.4f} | "
              f"Test Loss: {test_loss:.4f} | "
              f"Train Acc: {train_acc:.4f} | "
              f"Test Acc: {test_acc:.4f}")
        
        train_losses.append(train_loss)
        train_accs.append(train_acc)
        test_losses.append(test_loss)
        test_accs.append(test_acc)
        
    if plot:
        plot_metrics(train_losses, test_losses, train_accs, test_accs)
\end{lstlisting}

\end{document}


